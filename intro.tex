%%%%%%%%%%%%%%%%%%%%%%%%%%%%%%%%%%%%%%%%%%%%%%%%%%%%%%%%%%%%%%%%%%%%%%%%%%%%%%%%
% Keep each line less than 80 characters.
%%%%%%%%%%%%%%%%%%%%%%%%%%%%%%%%%%%%%%%%%%%%%%%%%%%%%%%%%%%%%%%%%%%%%%%%%%%%%%%%
\chapter{引\quad 言}

%^C^C Overview of Large Language Models.
% \par Increased computational power, advanced machine-learning model
% \citep{attention}, and access to very large-scale data have led to a
% significant transition to the emergence of \textbf{Large Language Models}
% \citep{overviewllm, llmsurvey}. Trained with expansive and diverse data, pretrained 
% large language models have demonstrated remarkable capability to 
% understand and generate human-like text, making them suitable for 
% various Natural Language Processing (NLP) tasks\citep{nlp}.
% Further Supervised Fine-Tuning (SFT) \citep{sft} equips such models with 
% ability to follow natural language instructions,
% making them powerful tools for a wide range 
% of applications. Large language models, such as ChatGPT \citep{chatgpt},
% LLama \citep{llama}, and Gemini \citep{gemini}, because of their
% rich domain-specific knowledge, have been employed be engineers and
% researchers to automate workflows across various domains.

\par 算力的增长, 先进的机器学习模型\citep{attention}, 以及对大规模数据的访问,
促成了\textbf{大语言模型}的出现\citep{overviewllm, llmsurvey}。
预训练的大语言模型通过大规模多样化的数据训练,展现了理解和生成类人文本的能力,
使其适用于各种自然语言处理(Natural Language Processing, NLP)任务\citep{nlp}。
进一步的监督微调(SFT)\citep{sft}使这些模型能够遵循自然语言指令,使它们
成为各种应用的强大工具。大语言模型,如ChatGPT \citep{chatgpt}, LLama 
\citep{llama}, 和Gemini \citep{gemini}, 因其丰富的领域特定知识,被工程
师和研究人员用于自动化各个领域的工作流程。

%^C^C Narrow down;
%^C^C Jump to the point of automated debugging techniques as a whole.
% \par Repairing bugs can be a time-consuming and daunting task
% for even a expert experienced in debugging \citep{incorrectfix}.
% Studies showed that
% debugging activities often accounts for about 50\% of overall 
% development cost of software products \citep{reversibledebug, omnitabledebug}.
% Many factors contribute to the cost of debugging, but the most influential
% ones are the manual efforts required to localize and fix faults 
% \citep{asrsurvey}. Researchers have developed a class of approach,
% namely \textbf{Automated Program Repair}
% \footnote{also known as Automatic Software Repair}, whose key idea is to
% try to automatically localize faults in software systems and produce 
% an actual patch to eliminate them 
% \citep{asrsurvey, asrbib, autobugfixsurvey}.
% Before the era of large language models, automated program repair (APR)
% is implemented by Genetic Programming \citep{genprog}, 
% random search \citep{randomsearch},
% or applying fix patterns/templates \citep{parsystem, tbar}.
% However, empirical studies on automated program repair tools 
% suggest that they struggle to help programmers with bug understanding
% \citep{autodebughelpful, debugbench, expectationonfl}, 
% (some) depends on manually crafted templates,
% and tend to overfit the test suite\citep{le2018overfitting, overfitinapr, invalidator}
% (i.e. the generated patch violates correctness specification or 
% maintains erroneous behavior).
% To overcome the difficulties to debug real-world systems and generate 
% plausible patch, large language models have been employed for automated
% patch synthesis \citep{fusellm4apr, autobugfixsurvey}. Thanks to their
% context-awareness, proficient coding skills \citep{csllm, codex},
% and zero/few short learning capabilities \citep{fewshotlearner},
% code language models have driven multiple fault localization 
% and APR tools and  demonstrated high effectiveness in fixing 
%  software defects and vulnerabilities
% \citep{securemind, llm4testfreefl, vulmaster} 
% across various APR benchmarks \citep{defects4j, swebench, cvedataset}.
%^C^C Stop here. do not talk about limitations of LLM-driven APR.

\par 修复软件中的错误对于经验丰富的专家来说也是一项耗时且艰巨
的任务\citep{incorrectfix}。
研究表明,调试活动通常占软件产品整体开发成本的约
50\%\citep{reversibledebug, omnitabledebug}。
许多因素导致了调试的成本,但最有影响力的因素是定位和修复错误所
需的人工努力\citep{asrsurvey}。
研究人员开发了一类方法,即\textbf{自动程序修复}
\footnote{也称为自动软件修复},其核
心思想是尝试自动定位软件系统中的错误并生成实际的补丁来消除
它们\citep{asrsurvey, asrbib, autobugfixsurvey}。
在大语言模型时代之前,自动程序修复(APR)是通过遗传
编程\citep{genprog}、随机搜索\citep{randomsearch}或应用修
复模式/模板\citep{parsystem, tbar}来实现的。
然而,对自动程序修复工具的实证研究表明,它们在帮助程序员理解错误方面存
在困难\citep{autodebughelpful, debugbench, expectationonfl},有些
依赖于手工制作的模板,并且倾向于过拟合测
试套件\citep{le2018overfitting, overfitinapr, invalidator}(
即生成的补丁违反了正确性规范或保持了错误行为)。
为了克服调试现实世界系统和生成合理补丁的困难,大语言模型被用于自动补
丁合成\citep{fusellm4apr, autobugfixsurvey}。得益于它们的上下文
感知能力、熟练的编码技能\citep{csllm, codex}和零/少量学习
能力\citep{fewshotlearner},代码语言模型已经驱动了多个错误定位和
APR工具,并在各种APR基准测试\citep{defects4j, swebench, cvedataset}中
展示了修复软件缺陷和漏洞的高效性
\citep{securemind, llm4testfreefl, vulmaster}。

%^C^C Narrow down;
%^C^C What about automated repairment of memory bugs?
%^C^C existing works, challenges (why difficult), etc.
%^C^C How does LLM-driven APR tools perform?
% \par\texttt{C} and \texttt{C++} are popular system programming languages 
% which do not check memory accesses during execution, causing
% memory bugs to slip in. Memory bugs, such as stack/heap overflow,
% use-after-free, and double free, are common causes of software vulnerabilities 
% and can lead to severe consequences. Moreover, memory bugs 
% are notoriously hard to localize; for a single memory error may not
% have immediately visible program misbehaviors (e.g. process crash or abort).
%  Dynamic analysis and tracing tools\citep{asan, valgrind} 
% therefore have been implemented to 
% detect memory errors at runtime, and abort program execution  
% with meaningful error messages when a memory error occur.
% Nevertheless, these tools require instrumentation at compile time and
% slow down program execution by more than 3x, limiting their
% applications during debugging or testing.
% Large language models have been employed to repair 
% memory bugs and consequent CVEs in \texttt{C} and \texttt{C++} programs. 
% Through preliminary literature review, we found that existing works
% addressing automated vulnerability repairment, including 
% VulRepair \citep{vulrepair}, VulAdvisor \citep{vuladvisor}, 
% VulMaster \citep{vulmaster},
% and Enhanced automated code vulnerability repair \citep{vulrepairusingllm}
% requires the computation-intense task of fine-tuning the 
% backbone language model. Luo et al. integrated Control Flow Graph
% information into their prompt, enabling code LMs to fix 
% vulnerabilities in real-world code \citep{prompt4vulrepair}.
% These works set the basis of automated program repair 
% \textit{when the root-cause of the bug is known}
% (i.e. the vulnerability type is known, and the vulnerable code is localized).
% To our knowledge, this is the first work to try to automate
% memory access bug localization at a fine-grained
% granularity and a repository level, which is neglected by 
% previous research on vulnerability repair
% but can potentially free developers from the
% tedious workflow of reviewing bug reports, reproducing memory errors
% and fix bugs.

\par \texttt{C}和\texttt{C++}是流行的系统编程语言,在执行期间不
检查内存访问,导致内存错误的出现。内存错误,如栈/堆
溢出 (stack/heap overflow)、使用后
释放 (use-after-free) 和双重释放 (double-free),是软
件漏洞的常见原因,并可能导致严重后果。此外,内存错误 通常认为
难以定位;单个内存错误可能没有立即可见的程序异常行为(例如进程崩溃或中止)。
因此,动态分析和跟踪工具\citep{asan, valgrind}已经被实现来在运行时检测
内存错误,并在发生内存错误时使用有意义的错误消息中止程序执行。然而,这些
工具需要在编译时进行插桩,并将程序执行速度降低3倍以上,限制了它
们在调试或测试期间的应用。大语言模型已被用于修复\texttt{C}和\texttt{C++}程序
中的内存错误和随之而来的CVE (Common Vulnerabilities and Exposures)。通过初
步文献回顾,我们发现现有解决自动化漏洞修复的方法,包括
VulRepair \citep{vulrepair}, VulAdvisor \citep{vuladvisor}, VulMaster \citep{vulmaster}, 
和增强自动代码漏洞修复\citep{vulrepairusingllm}需要计算密集型任务来微调 (fine-tune) 骨干语言
模型。Luo等人将控制流图信息集成到他们的提示中,使代码LM能够修复现实世界
代码中的漏洞\citep{prompt4vulrepair}。这些工作为自动程序修复
奠定了基础\textit{当错误的根本原因已知时}(即漏洞类型已知,并且易受攻击的
代码已定位)。据我们所知,这是第一项尝试在细粒度粒度和仓库级别自动化内存访问错
误定位的工作,这在先前关于漏洞修复的研究中被忽视,但可以潜在地使开发人员摆脱审
查错误报告、重现内存错误和修复错误的繁琐工作流程。

%^C^C Problem formulation (TO BE DETERMINED, NEED LITERATURE REVIEW).
% \paragraph{Problem Formulation} Given a \texttt{C,C++} 
% project repository and its bug report consisting of program input,
% and stack trace when program \textbf{crashed}
% \footnote{Including function name and line number}, can we
% utilize coding LLMs or agentic systems to automate bug localization
% and resolution, assuming the following experimental constraints?

% \begin{itemize}
%     \item Without any additional information 
%      from address sanitizer or similar memory bug detectors.
%     Because of high runtime overhead of these sanitizers,
%     they are not likely to be included in the release
% 	  version of the buggy application.

%    \item Without finetuning existing models or train 
% 	  new language models, i.e. only prompting 
%     published code LM.
%     This is needed to overcome 
% 	  the high computation cost of existing vnulreability 
%     repairment tools.
% \end{itemize}

% We believe this scenario is realistic enough for today's
% mobile-application developers who spend a considerable 
% amount of time reviewing bug reports from end-users
% who unually install a release-version of the buggy
% application, and have no access to the source code or
% debugging tools.

\paragraph{问题表述} 
给定一个\texttt{C,C++}项目仓库及其错误报告,其中包含程序输入和
程序\textbf{崩溃}时的堆栈跟踪\footnote{包括函数名称和行号},我们能否
利用编码LLM或智能系统在以下实验约束条件下自动化错误定位和解决?

\begin{itemize}
	\item 不使用地址消毒器 (Address Sanitizer\citep{asan}) 
			或类似内存错误检测器\citep{valgrind}的任何其他信息
			。由于这些消毒器的高运行时开销,它们不太可能包含在有缺陷应用程序的发布版本中。
    \item 不微调现有模型或训练新的语言模型,即仅提
			示发布的代码语言模型。这是为了克服现有漏洞修复工具的高计算成本。
\end{itemize}

我们相信这个场景对于当今的移动应用程序开发人员来说足够现实,
他们花费大量时间审查来自最终用户的错误报告,这些用户通常安装
了有缺陷应用程序的发布版本,并且无法访问源代码或调试工具。

% We summarize our contributions as follows:
% + We provide a Systematic Literature Review of articles related to LLM-based
% automatic program repair (LLM4APR) since 2021; 
\par 我们归纳我们的贡献如下:
\begin{itemize}
    \item 我们详尽检索了2021年, 有关"基于大语言模型的自动程序修复方法"
    (LLM4APR)的文献,
    并提供了系统文献综述 (Systematic Literature Review, SLR);
\end{itemize}
