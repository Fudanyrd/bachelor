%%%%%%%%%%%%%%%%%%%%%%%%%%%%%%%%%%%%%%%%%%%%%%%%%%%%%%%%%%%%%%%%%%%%%%%%%%%%%%%%
% Please Keep each line less than 80 characters.
%
% requires: data/bibtex/slr.bib
% requires packge: natbib,
%%%%%%%%%%%%%%%%%%%%%%%%%%%%%%%%%%%%%%%%%%%%%%%%%%%%%%%%%%%%%%%%%%%%%%%%%%%%%%%%
\chapter{引 言}

% We summarize our contributions as follows:
% + We provide a Systematic Literature Review of articles related to LLM-based
% automatic program repair since 2021; 
\par 我们归纳我们的贡献如下:
\begin{itemize}
    \item 我们详尽检索了2021年, 有关"基于大语言模型的自动程序修复方法"的文献,
    并提供了系统文献综述 (Systematic Literature Review, SLR);
\end{itemize}

\chapter{相 关 工 作}

% \par We present a Systematic Literature Review (SLR) including articles
% published in January 2021 and Feburary 2026 that focuses on the use of
% LLM based solutions to Automated Program Repairs (APR).
% The SLR follows the methdology proposed by Kitchenham et al.
% \citep{slrguidese, segress}, used in many SE-related SLRs
% \citep{dl4defence, ml4se, llm4se2}. Following the guidelines provided by
% Kitchenham et al., our methodology include three main steps:
% planning the review (i.e. Section~\ref{sec:search}), conducting the review
% (i.e. Section~\ref{sec:selection}), and analyzing the basic review results 
% (i.e. Section~\ref{sec:analysis}).
\par 我们进行了一项系统文献综述(SLR),涵盖了2021年1月至2026年2月期间发表的文章,
重点关注基于大型语言模型(LLM)的自动程序修复(APR)解决方案的使用情况。
这项SLR遵循了Kitchenham等人提出的方法论\citep{slrguidese, segress},该方法已被许
多与软件工程相关的SLR使用\citep{dl4defence, ml4se, llm4se2}。
根据Kitchenham等人提供的指南,我们的方法论包括三个主要步骤:
规划审查(即第\ref{sec:search}节),进行审查(即第\ref{sec:selection}节),以及
分析基本审查结果(即第\ref{sec:analysis}节)。

% \section{Search Strategy}\label{sec:search}
\section{检索方法}\label{sec:search}

% \par We employed the \textbf{Precise Search Strategy} \citep{optimalsearch}
% for article search, i.e., we crafted the search string to maximize the relevance 
% of resulting studies. More specifically, we followed the following three steps
% in order to establish a set of relevant studies:
%
% \begin{enumerate}
%  \item Conduct an automated search based on our crafted search string;
%  \item Screen the title and abstract of all articles and filter by
%        inclusion/exclusion criteria;
%  \item Conduct snowballing search on the result of previous step.
% \end {enumerate}

\par 我们采用了\textbf{精确搜索策略} \citep{optimalsearch}进行文章检索,
即我们精心设计了搜索字符串,以最大化结果研究的相关性。更具体地说,
我们遵循以下三个步骤来建立一组相关研究:

\begin{enumerate}
    \item 基于我们设计的搜索字符串进行自动化搜索;
    \item 筛选所有文章的标题和摘要,并根据包含/排除标准进行过滤;
    \item 对上一步的结果进行滚雪球式搜索 (Snowballing Search)。
\end{enumerate}

% \par Our search string needs to combine three set of keywords: 
% one pertaining to software defects, one related to large language models,
% last about the task definition. The complete set of search keywords is
% as follows:
\par 我们的搜索字符串需要结合三组关键词:一组与软件缺陷相关,一组与大型语言模型相关,
最后一组与任务定义相关。完整的搜索关键词如下:
\begin{itemize}
  % \item \textit{Keywords related to LLMs}:
  \item \textit{与大语言模型相关的关键词}:
  large language model, LLM, GPT, CodeX, agent
  % \item \textit{Keywords related to Task Definition}:
  \item \textit{与任务定义相关的关键词}:
  repair, resolve, reproduce, fix, localize
  % \item \textit{Keywords related to software defects}:
  \item \textit{与软件缺陷相关的关键词}:
  bug, defect, vulnerability, crash
\end{itemize}

% \par It is important to note that the list of keywords related to 
% LLMS we setup includes \textit{agent}, that does not seem to be
% necessarily related to LLMs. The reason for this is that we observe 
% recent advancement in agentic technologies \citep{agents}
% and their applications in Software Engineering
% \citep{agent4bugfixempirical,agenticbugreproductioneffective,langgraphbugfix},
%  and that we want to avoid omitting
% articles related to our research as much as possible.
\par 需要注意的是,我们设置的与LLMs相关的关键词列表中包括\textit{agent},
这似乎并不一定与LLMs相关。
原因是我们观察到最近在代理技术方面的进展\citep{agents}以及它们在软件
工程中的应用
\citep{agent4bugfixempirical,agenticbugreproductioneffective,langgraphbugfix},
我们希望尽可能避免遗漏与我们的研究相关的文章。

%%%%%%%%%%%%%%%%%%%%%%%%%%%%%%%%%%%%%%%%%%%%%%%%%%%%%%%%%%%%%%%%%%%%%%%%%%%%%%%%
% \paragraph{Search Databases}. After determining the search strings, we
% employed the SLR Tool\citep{slrtool} and conducted
% an automated search across four widely used databases
% \footnote{ACM Digital Library, arXiv, IEEE Xplore, Springer}, 
% which cover most published
% articles and preprints of under-review articles. Since we observe that 
% most LLM4SE articles are published after 2021, we focus our search on articles
% published from that year onward. The search results from each database were
% merged and deduplicated with SLR Tool\citep{slrtool}. Specifically, we obtained
% 373 articles from ACM Digital Library, 213 articles from arXiv, 9 articles
% from IEEE Xplore, and 1187 articles from Springer.
\paragraph{搜索数据库}. 确定搜索字符串后,我们使用SLR工具\citep{slrtool}在四个广泛
使用的数据库
\footnote{ACM Digital Library, arXiv, IEEE Xplore, Springer}上进行了自动化搜索,
这些数据库涵盖了大多数已发表的文章和正在审核的预印本文章。由于我们观察到大多数LLM4SE文
章是在2021年之后发表的,我们将搜索重点放在从那年开始发表的文章上。
每个数据库的搜索结果都通过SLR工具\citep{slrtool}进行了合并和去重。
具体来说,我们从ACM Digital Library获得了373篇文章,从arXiv获得了213篇文章,
从IEEE Xplore获得了9篇文章,从Springer获得了1187篇文章。

% \section{Study Selection}\label{sec:selection}
\section{研究选择}\label{sec:selection}

% \section{Data Extraction and Analysis}\label{sec:analysis}
\section{数据提取与分析}\label{sec:analysis}

